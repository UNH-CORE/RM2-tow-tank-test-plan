%\documentclass[pdf,12pt,report,strict]{SANDreport}
\documentclass{article}
\usepackage{subfigure}
\usepackage[latin1]{inputenc}
\usepackage{amsmath}
\usepackage{amsfonts}
\usepackage{amssymb}
\usepackage{graphicx}
\usepackage{todonotes}

\author{P. Bachant, M. Wosnik, V. Neary}
\title{Test Plan: UNH RM2 Tow Tank Experiment}

%\SANDnum{NUMBER}		% e.g. \SANDnum{SAND2006-0420}
%\SANDprintDate{DATE}	% Month, year
%\SANDauthor{AUTHORS}	% One line, separated by commas

\begin{document}

\maketitle
%\tableofcontents

\section{Introduction}

\todo[inline]{Write an introduction...}

\subsection{Objectives}

\begin{enumerate}

	\item Acquire a Reynolds number independent dataset for the DOE RM2 cross-flow
	turbine.

		\begin{enumerate}
			\item Design and build an appropriate turbine model.
		
			\item Acquire full device performance curves (power and drag coefficients) at
			multiple $Re$, to attempt to find convergence.
		  
			\item Once convergence is found, acquire another performance curve at that
			Reynolds number $Re_0$.
		
			\item If time permits, acquire a detailed wake flow map in the near-wake at
			$Re_0$ using acoustic Doppler velocimetry.
		\end{enumerate}
	
	\item Measure the effects of strut drag on RM2 performance.
	
	\begin{enumerate}
		\item Measure the parasitic torque from support struts by rotating the turbine
		in still water. \label{obj-parasitic_torque}
		
		\item Redo task \ref{obj-parasitic_torque} with cylindrical tubes slid over
		the struts to significantly increase drag.
		
		\item Acquire a performance curve at $Re_0$ with the high-drag struts. 
	\end{enumerate}
	
\end{enumerate}


\section{Experimental setup and methods}

\subsection{Turbine model}

The turbine is to be a 1:6 scale model of the DOE RM2 rotor.

\subsubsection{Geometric requirements}

Turbine geometry is to be scaled from the RM2 design report \cite{Barone2011}.

\begin{table}[ht]
\centering
\begin{tabular}{l|l|l}
   & Full-scale & Model (1:6) \\
\hline 
Diameter (m)   & 6.45 & 1.075 \\ 
Height (m)     & 4.84 & 0.807 \\ 
Blade root chord (m) & 0.400 & 0.067 \\ 
Blade tip chord (m)  & 0.240 & 0.040 \\ 
Blade profile & NACA 0021 & NACA 0021 \\ 
Blade mount & 1/2 chord & 1/2 chord \\ 
Blade pitch (deg.) & 0.0 & 0.0 \\ 
Strut profile & NACA 0021 & NACA 0021 \\ 
Strut chord (m) & 0.360 & 0.060 \\ 
Shaft diameter (m) &  &  \\ 
\end{tabular}
\caption{RM2 turbine geometric parameters.}
\end{table}

The shaft diameter has still not been chosen.


\subsection{Facility and instrumentation}

Experiments will be performed in the UNH tow/wave tank. The turbine will be
mounted in a frame built from NACA 0020 struts, attached to the tow carriage by
four linear bearings, which transfer all streamwise force to a pair of S-beam
load cells. The turbine shaft RPM will be controlled by a servo motor system,
which allows prescription of the turbine tip speed ratio. The load torque will
be measured by an inline rotary torque transducer and a load cell mounted at a
fixed distance from the servo motor, providing a redundant measurement. Turbine
shaft angle will be measured using the servo drive's emulated encoder output,
set to $10^5$ counts per turbine shaft revolution. Carriage speed, and therefore
inflow velocity will be measured using a linear encoder with 10 $\mu$m
resolution.


\section{Research deliverables}

A full performance curve for the turbine will consist of 20--30 tows at varying
tip speed ratio. Each of these tows will produce raw data for the turbine
torque, drag force on the submerged equipment, turbine shaft angle, carriage
speed, and Vectrino velocity measurements. Raw data files will be saved from our
data acquisition device at a sample sate of 2 kHz, from the motion controller at
1 kHz, and from the Vectrino at 200 Hz. All three devices will be triggered by
the motion controller at the beginning of each run so that they are relatively
synchronized.

Tare torque and drag runs will also be performed to measure the shaft bearing
friction torque and turbine mounting frame drag, respectively. These data will
be similar to the turbine performance data, omitting torque measurements for the
tare drag runs and vice versa.

Raw and processed data, along with the processing and plotting code, will be
hosted on both figshare and GitHub. Figshare provides a digital object
identifier (DOI) and persistent URL for each dataset, while GitHub provides a
mechanism for updating the code and processed data, both by the investigators
and third parties. Raw data will be in HDF5 format, processed data in comma
separated value (CSV), and processing code in Python.

\bibliography{../../../../../Library/Library}
\bibliographystyle{plain}

\end{document}